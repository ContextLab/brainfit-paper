\documentclass{article}
\usepackage[utf8]{inputenc}
\usepackage{geometry}
\usepackage[sort]{natbib}
\usepackage{pxfonts}
\usepackage{graphicx}
\usepackage{setspace}
\usepackage{hyperref}
\usepackage{lineno}
\usepackage{authblk}

\doublespacing
\linenumbers


\title{Fitness tracking reveals task-specific memory effects of exercise}
\author[1]{Jeremy R. Manning}
\author[1,2]{Gina M. Notaro}
\author[1]{Esme Chen}
\author[1]{Paxton C. Fitzpatrick}
\affil[1]{Dartmouth College, Hanover, NH}
\affil[2]{Lockheed Martin, Bethesda, MD}

\begin{document}
\maketitle

\begin{abstract}
Physical excercise can benefit both physical and mental well-being.  Different forms of exercise (i.e., aerobic versus anaerobic; running versus walking versus swimming versus yoga; high-intensity interval trainiing versus endurance workouts; etc.) impact physical fitness in different ways.  For example, running may substantially impact leg and heart strength but only moderately impact arm strength. We hypothesized that the mental benefits of exercise might be similarly differentiated.  We focused specifically on how different forms of exercise might affect different aspects of memory.  To test our hypothesis, we collected nearly a century's woth of fitness data (in aggregate).  We then asked participants to engage in a battery of memory tasks that tested their short and long term episodic, semantic, and spatial memory.  We found that ??.
\end{abstract}

\section*{Introduction and overview}

\section*{Acknowledgements}


\section*{Data and code availability}
All analysis code and data used in the present manuscript may be found \href{https://github.com/ContextLab/brainfit-paper}{\underline{here}}.

\bibliographystyle{apa}
\bibliography{CDL-bibliography/cdl}
\end{document}
