\documentclass[10pt]{article}
\usepackage[utf8]{inputenc}
\usepackage{geometry}
\usepackage[sort]{natbib}
\usepackage{pxfonts}
\usepackage{graphicx}
\usepackage{setspace}
\usepackage{hyperref}
\usepackage{lineno}
\usepackage{authblk}

\doublespacing
\linenumbers


\title{Fitness tracking reveals task-specific associations between
  memory, mental health, and exercise}
\author[1, $\star$]{Jeremy R. Manning}
\author[1,2]{Gina M. Notaro}
\author[1]{Esme Chen}
\author[1]{Paxton C. Fitzpatrick}
\affil[1]{Dartmouth College, Hanover, NH}
\affil[2]{Lockheed Martin, Bethesda, MD}
\affil[$\star$]{Address correspondence to jeremy.r.manning@dartmouth.edu}

\begin{document}
\maketitle

\begin{abstract}
  Physical exercise can benefit both physical and mental well-being.
  Different forms of exercise (i.e., aerobic versus anaerobic; running
  versus walking versus swimming versus yoga; high-intensity interval
  training versus endurance workouts; etc.) impact physical fitness
  in different ways.  For example, running may substantially impact
  leg and heart strength but only moderately impact arm strength. We
  hypothesized that the mental benefits of exercise might be similarly
  differentiated.  We focused specifically on how different forms of
  exercise might related to different aspects of memory and mental
  health.  To test our hypothesis, we collected nearly a century's
  worth of fitness data (in aggregate).  We then asked participants to
  fill out surveys asking them to self-report on different aspects of
  their mental health.  We also asked participants to engage in a
  battery of memory tasks that tested their short and long term
  episodic, semantic, and spatial memory.  We found that participants
  with similar exercise habits and fitness profiles tended to also
  exhibit similar mental health and task performance profiles.
\end{abstract}

\section*{Introduction}
Engaging in physical activity (exercise) can improve our physical
fitness by increasing muscle strength~\citep{RogeEvan93, Lind79,
  CranEtal13, Knut07}, increasing bone density~\citep{ChilEtal12,
  BassRams94, LaynNels99}, increasing cardiovascular
performance~\citep{MaioEtal00, PollEtal00}, increasing lung
capacity~\citep{LazoEtal16}~\citep[although see][]{RomaEtal16},
increasing endurance~\citep{WilmKnut03}, and more.  Exercise can also
improve mental health~\citep{Ragl90, MikkEtal17, TaylEtal85,
  DeslEtal09, Call04, PaluSchw00, BassSuzu17} and cognitive performance~\citep{ChanEtal12b,
  BrisEtal02, EtniEtal06, BassSuzu17}.

The physical benefits of exercise can be explained by stress-responses
of the affected body tissues. For example, skeletal muscles that are
taxed during exercise exhibit stress responses~\citep{MortEtal09} that
can in turn affect their growth or atrophy~\citep{SchiEtal13}.  By
comparison, the benefits of exercise on mental health are less direct.  For
example, one hypothesis is that exercise leads to specific
physiological changes, such as increased aminergic synaptic
transmission and endorphin release, which in turn act on
neurotransmitters in the brain~\citep{PaluSchw00}.

Speculatively, if different exercise regimens lead to different
neurophysiological responses, one might be able to map out a spectrum
of signalling and transduction pathways that are impacted by a given
type, duration, and intensity of exercise in each brain region.  For
example, prior work has shown that exercise increases acetylcholine
levels, starting in the vicinity of the exercised
muscles~\citep{ShoeEtal97}.  Acetylcholine is thought to play an
important role in memory formation~\citep[e.g., by modulating specific
synaptic inputs from entorhinal cortex to the hippocampus, albeit in
rodents;]{PalaEtal21}.  Given the central role of these medial
temporal lobe structures play in memory, changes in acetylcholine
might lead to specific changes in memory formation and retrieval.

In the present study, we hypothesize that (a) different exercise regimens will
have different, quantifiable impacts on cognitive performance and
mental health, and that (b) these impacts will be consistant across
individuals.  To this end, we collected a year of fitness tracking
data from each of 113 participants.  We then asked each participant to
fill out a brief survey in which they self-evaluated several aspects
of their mental health.  Finally, we ran each participant through a
battery of memory tasks, which we used to evaluate their memory
performance along several dimensions.  We examined the data for
potential associations between memory, mental health, and exercise.

\section*{Results}
\begin{itemize}
  \item exploratory analysis (correlations)
    \begin{itemize}
    \item Memory-memory
    \item fitness-fitness
    \item survey-survey
    \item (fitness + survey)-memory
    \end{itemize}
  \item predictive analysis (regressions)
    \begin{itemize}
    \item Predict memory performance on held-out task from other tasks
    \item Predict memory performance on each task using fitness data
      \item Predict memory performance on each task using survey data
      \end{itemize}
    \item Reverse correlations: look at recent changes versus baseline trends
      \begin{itemize}
      \item Fitness profile that predicts performance on each task (barplots + timelines)
      \item Fitness profile for each survey demographic (barplots + timelines)
        \begin{itemize}
          \item Select out mental health demographics (based on meds, stress levels)
          \end{itemize}
        \end{itemize}
  \end{itemize}

  \section*{Discussion}
  \begin{itemize}
  \item summarize key findings
  \item correlation versus causation
         \item what can vs. can't we know?  we can identify correlations, but not causal direction-- e.g. we cannot know whether exercise \textit{causes} mental changes versus whether people with particular neural profiles might tend to engage in particular exercise behaviors.  that being said, we \textit{can} separate out baseline tendencies (e.g., how people tend to exercise in general) versus recent changes (e.g., how they happened to have exercised prior to the experiment).
  \item related work (exercise/memory, exercise/mental health), what this study adds
    \item future direction: towards customized physical exercise recommendation engine for optimizing mental health and mental fitness
    \end{itemize}

    \section*{Methods}

    We ran an online experiment using the Amazon Mechanical Turk
    platform.  We collected data about each participant's fitness and
    exercise habits, a variety of self-reported measures concerning their
    mental health, and about their performance on a battery of memory
    tasks.  We mined the dataset for potential associations between
    memory, mental health, and exercise.
    
\subsection*{Experiment}
\subsubsection*{Participants}
We recruited experimental participants by posting our experiment as a
Human Intelligence Task (HIT) on the Amazon Mechanical Turk platform.
We limited participation to Mechanical Turk Workers who had been
assigned a ``Masters'' designation on the platform, given to workers
who score highly across several metrics on a large number of HITs,
according to a proprietary algorithm managed by Amazon.  We further
limited our participant pool to participants who self-reported that
they were fluent in English and regularly used a Fitbit fitness
tracker device.  A total of 160 workers accepted our
HIT in order to participate in our experiment.  Of these, we excluded all
participants who failed to log into their Fitbit account (giving us
access to their anonymized fitness tracking data), encountered
technical issues (e.g., by accessing the HIT using an incompatible browser, device, or
operating system), or who ended their participation prematurely,
before completing the full study.  In all, 113 participants remained
that contributed usable data to the study.

For their participation, workers received a base payment of \$5 per hour (computed in 15
minute increments, rounded up to the nearest 15 minutes), plus an
additional performance-based bonus of up to \$5.  Our recruitment
procedure and study protocol
were approved by Dartmouth's Committee for the Protection of Human Subjects.

\paragraph{Gender, age, and race.}
Of the 113 participants who contributed usable data, 77 reported their gender as female, 35 as
male, and 1 chose not to report their gender.  Participants ranged in
age from 19--68 years old (25\textsuperscript{th} percentile: 28.25
years; 50\textsuperscript{th} percentile: 32 years;
75\textsuperscript{th} percentile: 38 years).  Participants reported
their race as White (90 participants), Black or African American (11
participants), Asian (7 participants), Other (4 participants), and
American Indian or Alaska Native (3 participants).  One participant
opted not to report their race.

\paragraph{Languages.}
All participants reported that they were fluent in either 1 and 2
languages (25\textsuperscript{th} percentile: 1;
50\textsuperscript{th} percentile: 1; 75\textsuperscript{th}
percentile: 1), and that they were ``familiar'' with between 1 and 11
languages (25\textsuperscript{th} percentile: 1;
50\textsuperscript{th} percentile: 2; 75\textsuperscript{th}
percentile: 3).

\paragraph{Reported medical conditions and medications.}
Participants reported having and/or taking medications pertaining to the following medical conditions: anxiety or
depression (4 participants), recent head injury (2 participants), high
blood pressure (1 participant), bipolar (1 participant),
hypothyroidism (1 participant), and other unspecified medications (1
participant).  Participants reported their current and typical stress
levels on a Likert scale as very relaxed (-2), a little relaxed (-1),
neutral (0), a little stressed (1), or very stressed (2).  The
``current'' stress level reflected participants' stress at the time
they participated in the experiment.
Their responses
ranged from -2 to 2 (current stress: 25\textsuperscript{th} percentile: -2;
50\textsuperscript{th} percentile: -1; 75\textsuperscript{th}
percentile: 1; typical stress: 25\textsuperscript{th} percentile: 0;
50\textsuperscript{th} percentile: 1; 75\textsuperscript{th}
percentile: 1).  Participants also reported their current level of
alertness on a Likert scale as very sluggish (-2), a little sluggish
(-1), neutral (0), a little alert (1), or very alert (2).  Their
responses ranged from -2 to 2 (25\textsuperscript{th} percentile: 0;
50\textsuperscript{th} percentile: 1; 75\textsuperscript{th}
percentile: 2).

\paragraph{Residence and level of education.}
Participants reported their residence
as being located in the suburbs (36 participants), a large city (30
participants), a small city (23 participants), rural (14
participants), or a small town (10 participants).  Participants
reported their level of education as follows: College graduate (42
participants), Master's degree (23 participants), Some college (21
participants), High school graduate (9 participants), Associate's
degree (8 participants), Other graduate or professional school (5
participants), Some graduate training (3 participants), or Doctorate
(2 participants).

\paragraph{Reported water and coffee intake.}
Participants reported the number of cups of water and coffee they had
consumed prior to accepting the HIT.  Water consumption ranged from
0--6 cups (25\textsuperscript{th} percentile: 1;
50\textsuperscript{th} percentile: 3; 75\textsuperscript{th}
percentile: 4).  Coffee consumption ranged from 0--4 cups (25\textsuperscript{th} percentile: 0;
50\textsuperscript{th} percentile: 1; 75\textsuperscript{th}
percentile: 2).


\subsubsection*{Tasks}
\paragraph*{Intake survey.}
\paragraph*{Free recall.}
\paragraph*{Naturalistic recall.}
\paragraph*{Foreign language flashcards.}
\paragraph*{Spatial learning.}

\subsubsection*{Fitness tracking using Fitbit devices}

\subsubsection*{Processing Fitbit data}
\paragraph*{Raw metrics.}
\paragraph*{Comparing recent versus baseline measurements.}

\subsubsection*{Exploratory correlation analyses}
\paragraph*{Imputation and interpolation of missing data.}

\subsubsection*{Regression-based prediction analyses}

\subsubsection*{Reverse correlation analyses}




\section*{Acknowledgements}
We acknowledge useful discussions with David Bucci, Emily Glasser, Andrew Heusser, Abigail Bartolome, Lorie Loeb, Lucy Owen, and Kirsten Ziman.  Our work was supported in part by the Dartmouth Young Minds and Brains initiative.  The content is solely the responsibility of the authors and does not necessarily represent the official views of our supporting organizations.  This paper is dedicated to the memory of David Bucci, who helped to inspire the theoretical foundations of this work.  Dave served as a mentor and colleague on the project prior to his passing.


\section*{Data and code availability}
All analysis code and data used in the present manuscript may be found \href{https://github.com/ContextLab/brainfit-paper}{\underline{here}}.

\section*{Author contributions}
Concept: J.R.M.  Experiment implementation and data collection: G.M.N.
Analyses: G.M.N., E.C., P.C.F., and J.R.M.  Writing: J.R.M.

\section*{Competing interests}
The authors declare no competing interests.

\bibliographystyle{apa}
\bibliography{/Users/jmanning/CDL-bibliography/cdl}
\end{document}
