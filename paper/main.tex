\documentclass[10pt]{article}
\usepackage[utf8]{inputenc}
\usepackage{geometry}
\usepackage[sort]{natbib}
\usepackage{pxfonts}
\usepackage{graphicx}
\usepackage{setspace}
\usepackage{hyperref}
\usepackage{lineno}
\usepackage{authblk}

\doublespacing
\linenumbers


\title{Fitness tracking reveals task-specific associations between
  memory, mental health, and exercise}
\author[1, $\star$]{Jeremy R. Manning}
\author[1,2]{Gina M. Notaro}
\author[1]{Esme Chen}
\author[1]{Paxton C. Fitzpatrick}
\affil[1]{Dartmouth College, Hanover, NH}
\affil[2]{Lockheed Martin, Bethesda, MD}
\affil[$\star$]{Address correspondence to jeremy.r.manning@dartmouth.edu}

\begin{document}
\maketitle

\begin{abstract}
  Physical excercise can benefit both physical and mental well-being.
  Different forms of exercise (i.e., aerobic versus anaerobic; running
  versus walking versus swimming versus yoga; high-intensity interval
  trainiing versus endurance workouts; etc.) impact physical fitness
  in different ways.  For example, running may substantially impact
  leg and heart strength but only moderately impact arm strength. We
  hypothesized that the mental benefits of exercise might be similarly
  differentiated.  We focused specifically on how different forms of
  exercise might related to different aspects of memory and mental
  health.  To test our hypothesis, we collected nearly a century's
  woth of fitness data (in aggregate).  We then asked participants to
  fill out surveys asking them to self-report on different aspects of
  their mental health.  We also asked participants to engage in a
  battery of memory tasks that tested their short and long term
  episodic, semantic, and spatial memory.  We found that participants
  with similar exercise habits and fitness profiles tended to also
  exhibit similar mental health and task performance profiles.
\end{abstract}

\section*{Introduction}
Engaging in physical activity (exercise) can improve our physical
fitness by increasing muscle strength~\citep{RogeEvan93, Lind79,
  CranEtal13, Knut07}, increasing bone density~\citep{ChilEtal12,
  BassRams94, LaynNels99}, increasing cardiovascular
performance~\citep{MaioEtal00, PollEtal00}, increasing lung
capacity~\citep{LazoEtal16}~\citep[although see][]{RomaEtal16},
increasing endurance~\citep{WilmKnut03}, and more.  Exercise can also
improve mental health~\citep{Ragl90, MikkEtal17, TaylEtal85,
  DeslEtal09, Call04, PaluSchw00, BassSuzu17} and cognitive performance~\citep{ChanEtal12b,
  BrisEtal02, EtniEtal06, BassSuzu17}.

The physical benefits of exercise can be explained by stress-responses
of the affected body tissues. For example, skeletal muscles that are
taxed during exercise exhibit stress responses~\citep{MortEtal09} that
can in turn affect their growth or atrophy~\citep{SchiEtal13}.  By
comparison, the benefits of exercise on mental health are less direct.  For
example, one hypothesis is that exercise leads to specific
physiological changes, such as increased aminergic synaptic
transmission and endorphin release, which in turn act on
neurotransmitters in the brain~\citep{PaluSchw00}.

Speculatively, if different exercise regimens lead to different
neurophysiological responses, one might be able to map out a spectrum
of signalling and transduction pathways that are impacted by a given
type, duration, and intensity of exercise in each brain region.  For
example, prior work has shown that exercise increases acetylcholine
levels, starting in the vicinity of the exercised
muscles~\citep{ShoeEtal97}.  Acetylcholine is thought to play an
important role in memory formation~\citep[e.g., by modulating specific
synaptic inputs from entorhinal cortex to the hippocampus, albeit in
rodents;]{PalaEtal21}.  Given the central role of these medial
temporal lobe structures play in memory, changes in acetylcholine
might lead to specific changes in memory formation and retrieval.

In the present study, we hypothesize that (a) different exercise regimens will
have different, quantifiable impacts on cognitive performance and
mental health, and that (b) these impacts will be consistant across
individuals.  To this end, we collected a year of fitness tracking
data from each of 113 participants.  We then asked each participant to
fill out a brief survey in which they self-evaluated several aspects
of their mental health.  Finally, we ran each participant through a
battery of memory tasks, which we used to evaluate their memory
performance along several dimensions.  We examined the data for
potential associations between memory, mental health, and exercise.

\section*{Results}
\begin{itemize}
  \item exploratory analysis (correlations)
    \begin{itemize}
    \item Memory-memory
    \item fitness-fitness
    \item survey-survey
    \item (fitness + survey)-memory
    \end{itemize}
  \item predictive analysis (regressions)
    \begin{itemize}
    \item Predict memory performance on held-out task from other tasks
    \item Predict memory performance on each task using fitness data
      \item Predict memory performance on each task using survey data
      \end{itemize}
    \item Reverse correlations: look at recent changes versus baseline trends
      \begin{itemize}
      \item Fitness profile that predicts performance on each task (barplots + timelines)
      \item Fitness profile for each survey demographic (barplots + timelines)
        \begin{itemize}
          \item Select out mental health demographics (based on meds, stress levels)
          \end{itemize}
        \end{itemize}
  \end{itemize}

  \section*{Discussion}
  \begin{itemize}
  \item summarize key findings
  \item correlation versus causation
         \item what can vs. can't we know?  we can identify correlations, but not causal direction-- e.g. we cannot know whether exercise \textit{causes} mental changes versus whether people with particular neural profiles might tend to engage in particular exercise behaviors.  that being said, we \textit{can} separate out baseline tendencies (e.g., how people tend to exercise in general) versus recent changes (e.g., how they happened to have exercised prior to the experiment).
  \item related work (exercise/memory, exercise/mental health), what this study adds
    \item future direction: towards customized physical exercise recommendation engine for optimizing mental health and mental fitness
    \end{itemize}

\section*{Methods}
\subsection*{Experiment}
\subsubsection*{Participants}
\subsubsection*{Tasks}
\paragraph*{Intake survey.}
\paragraph*{Free recall.}
\paragraph*{Naturalistic recall.}
\paragraph*{Foreign language flashcards.}
\paragraph*{Spatial learning.}

\subsubsection*{Fitness tracking using Fitbit devices}

\subsubsection*{Processing Fitbit data}
\paragraph*{Raw metrics.}
\paragraph*{Comparing recent versus baseline measurements.}

\subsubsection*{Exploratory correlation analyses}
\paragraph*{Imputation and interpolation of missing data.}

\subsubsection*{Regression-based prediction analyses}

\subsubsection*{Reverse correlation analyses}




\section*{Acknowledgements}
We acknowledge useful discussions with David Bucci, Emily Glasser, Andrew Heusser, Avigail Bartolome, Lorie Loeb, Lucy Owen, and Kirsten Ziman.  Our work was supported in part by the Dartmouth Young Minds and Brains initiative.  The content is solely the responsibility of the authors and does not necessarily represent the official views of our supporting organizations.  This paper is dedicated to the memory of David Bucci, who helped to inspire the theoretical foundations of this work.  Dave served as a mentor and colleague on the project prior to his passing.


\section*{Data and code availability}
All analysis code and data used in the present manuscript may be found \href{https://github.com/ContextLab/brainfit-paper}{\underline{here}}.

\section*{Author contributions}
Concept: J.R.M.  Implementation: G.M.N.  Analyses: G.M.N., E.C., and P.C.F.  Writing: J.R.M.

\section*{Competing interests}
The authors declare no competing interests.

\bibliographystyle{apa}
\bibliography{/Users/jmanning/CDL-bibliography/cdl}
\end{document}
