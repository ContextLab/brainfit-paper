\title{cover letter}
%
% See http://texblog.org/2013/11/11/latexs-alternative-letter-class-newlfm/
% and http://www.ctan.org/tex-archive/macros/latex/contrib/newlfm
% for more information.
%
\documentclass[11pt,stdletter,orderfromtodate,sigleft]{newlfm}
\usepackage{hyperref, pxfonts, geometry}

  \setlength{\voffset}{0in}

\newlfmP{dateskipbefore=0pt}
\newlfmP{sigsize=20pt}
\newlfmP{sigskipbefore=10pt}
 
\newlfmP{Headlinewd=0pt,Footlinewd=0pt}
 
\namefrom{\vspace{-0.3in}Jeremy R. Manning}
\addrfrom{
	Dartmouth College\\
    %Department of Psychological \& Brain Sciences\\
    %HB 6207 Moore Hall\\
	Hanover, NH  03755}
 
\addrto{}
\dateset{\today}
 
\greetto{To the editors of \textit{eLife}:}


 
\closeline{Sincerely,}
%\usepackage{setspace}
%\linespread{0.85}
% How will your work make others in the field think differently and move the field forward?
% How does your work relate to the current literature on the topic?
% Who do you consider to be the most relevant audience for this work?
% Have you made clear in the letter what the work has and has not achieved?

\begin{document}
\begin{newlfm}
We have enclosed our manuscript entitled \textit{Fitness tracking
  reveals task-specific associations between memory, mental health,
  and exercise} to be considered for publication as a
\textit{Research Article}.  The manuscript reports new evidence that, just as different
forms of exercise affect \textit{physical} health in different ways,
different forms of exercise also affect cognitive performance and
mental health differently.

In our study, we collected a year of fitness tracking data from each
of 113 participants.  We also had the participants fill out surveys
asking them to self-report on different aspects of their mental
health, and to engage in a battery of memory tasks that assessed their
short and long term episodic, semantic, and spatial memory
performance.  We found that participants with similar exercise habits
and fitness profiles tended to also exhibit similar mental health and
memory performance profiles.  We view the study as laying a foundation
for future exercise interventions that target specific aspects of
cognitive performance and mental health.


We expect that this article will be of interest to researchers in a
broad range of areas, including fitness, memory, and mental health.

Thank you for considering our manuscript, and we hope you will find it suitable for publication in \textit{eLife}.


\end{newlfm}
\end{document}
